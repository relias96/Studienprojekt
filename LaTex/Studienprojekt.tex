\documentclass[a4paper,12pt, oneside]{article} % twoside für doppelseitigen ausdruck

\usepackage[utf8]{inputenc} % für Umlate
\usepackage[ngerman]{babel} % deutsches TOC
\usepackage[babel]{csquotes} 
\usepackage[authoryear]{natbib} % Zitate  Autor (Jahr), bzw. (Autor, Jahr)

\usepackage{url} % Für Links im Text
\usepackage{float} % H als feste Position
\usepackage[font=footnotesize,labelfont=bf]{caption} % für schöner formatierte Captions
\usepackage[shortlabels]{enumitem} % \begin{enumerate}[a)] für a) enumerate
\usepackage{graphicx} % für bessere Grafiken
\usepackage[onehalfspacing]{setspace} % 1,5 facher Zeilenabstand
\usepackage{makecell}

\usepackage{booktabs}
\usepackage{multirow}
\usepackage{underscore}

%Formatierung von Zitaten
%Doppelpunkt vor der Seitenzahl
%Zitate in () Klammern
%mehrere Zitate mit ; trennen
%a Autor und Jahr
%, Trennzeichen zwischen Autor und Jahr
%,~ Trennzeichen bei mehreren Jahresangaben
\bibpunct[:~]{(}{)}{;}{a}{,}{,~}

\setlength\parindent{0pt} % kein linker Einzug bei neuen pars
%\usepackage[left=3cm,right=2.5cm,top=3.0cm,bottom=3.0cm]{geometry} %Ränder %%%% PRINT
\usepackage[left=2.75cm,right=2.75cm,top=3cm,bottom=3cm]{geometry} %Ränder 
\usepackage[nottoc]{tocbibind}

%schoenere Kopf- und Fusszeilen, für ein und doppelseitig
\usepackage{fancyhdr} \pagestyle{fancy}
\renewcommand\headrulewidth{0.4pt}
\fancyhead[ER,OL]{} 
\fancyfoot[C]{\thepage}
\fancyhead[EL,OR]{\textsl{\leftmark}} % nur die section oben links (nicht subsect)

% eine seite ohne sectionsnamen oben sondern nur mit linie
\fancypagestyle{justline}{
\fancyhead[R]{}
\renewcommand\headrulewidth{0.4pt}
\fancyfoot[C]{\thepage}
\fancyhead[OL,ER]{}
\fancyhead[EL,OR]{} }

\usepackage{footnpag} %fussnoten werden auf jeder Seite einzeln gezaehlt

\addto{\captionsngerman}{%
  %\renewcommand*{\contentsname}{Inhalt}
  \renewcommand*{\listfigurename}{Abbildungsverzeichnis}
  \renewcommand*{\listtablename}{Tabellenverzeichnis}
  \renewcommand*{\figurename}{Abb.}
  \renewcommand*{\tablename}{Tab.}
}
\usepackage{hyperref} % Links im PDF
\usepackage{lineno} %Zeilennummern von \linenumbers bis \nolinenumbers

\usepackage{anyfontsize}
\usepackage[textsize=scriptsize]{todonotes} % todos
\setlength{\marginparwidth}{2cm} % wie weit die todos nach rechts dürfen
\title{Verfassen von Hausarbeiten mit LaTeX}
\author{}

\usepackage{wrapfig}

%-----------Head Ende------------------------------------
\begin{document}
\renewcommand{\arraystretch}{1.2} % entwas entzerrtere Tabellen
\begin{titlepage}

\begin{center}
  \bfseries
  \LARGE Universität Osnabrück
  \vskip.6in
  \Large Studienprojekt %Worum handelt es sich? Hausarbeit/ Essay/ Aufgabenblatt
  \vskip0.15in 
  \small Betreut durch
  %\vskip.15in
  \Large %Studienprojekt  %Welches Modul?
  \vskip.3in
  \small  Prof. Dr. Frank Hilker
  \vskip.3in
  \small Wintersemester 2023 %Welches Semester
  \vskip.5in
  \vfill
  \rule{\textwidth}{0.25mm}
  \vskip0.25cm
  \LARGE Host-Parasitoid Interactions
  \vskip0.cm
  \rule{\textwidth}{0.25mm}
  \vskip1in
  
  \vfill
 \small verfasst von 
 \vskip.5in
\end{center}


\begin{minipage}{1\textwidth}
\begin{center}
\bfseries\large 
Robin  \\ Elias \\
\vskip.1in
968504\\ % Matrikelnummer
\vskip.1in
\small \textit{Umweltsystemwissenschaft Monobachelor} \\

\end{center}
\end{minipage}
\hfill


\vfill
\begin{center}
    Abgegeben am \today
\end{center}
\end{titlepage}

\thispagestyle{justline}
\setcounter{page}{1} % Seite eins ist hier
\setcounter{tocdepth}{2} % tocdepth 2 enthält bis zu subsection, 3 bis zu subsubsections
\tableofcontents
\listoffigures
\newpage

%%%%%%%%%%%%%%%%%
%Ab hier beginnt das Dokument
%%%%%%%%%%%%%%%%%





\newpage
\thispagestyle{justline}
{\small
\bibliographystyle{apalike-german}
%\bibliography{xyz} %hier den Namen der .bib Datei 
}


\end{document}