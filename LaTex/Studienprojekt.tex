\documentclass[a4paper,12pt, twoside]{article} % twoside für doppelseitigen ausdruck

\usepackage[utf8]{inputenc} % für Umlate
\usepackage[ngerman]{babel} % deutsches TOC
\usepackage[babel]{csquotes} 
\usepackage[authoryear]{natbib} % Zitate  Autor (Jahr), bzw. (Autor, Jahr)

\usepackage{url} % Für Links im Text
\usepackage{float} % H als feste Position
\usepackage[font=footnotesize,labelfont=bf]{caption} % für schöner formatierte Captions
\usepackage[shortlabels]{enumitem} % \begin{enumerate}[a)] für a) enumerate
\usepackage{graphicx} % für bessere Grafiken
\usepackage[singlespacing]{setspace} % 1,5 facher Zeilenabstand
\usepackage{makecell}

\usepackage{booktabs}
\usepackage{multirow}
\usepackage{underscore}
\usepackage{amsmath}
\usepackage{array}
\usepackage{romannum}
\usepackage{parskip}

%Formatierung von Zitaten
%Doppelpunkt vor der Seitenzahl
%Zitate in () Klammern
%mehrere Zitate mit ; trennen
%a Autor und Jahr
%, Trennzeichen zwischen Autor und Jahr
%,~ Trennzeichen bei mehreren Jahresangaben
\bibpunct[:~]{(}{)}{;}{a}{,}{,~}

\setlength\parindent{0pt} % kein linker Einzug bei neuen pars
%\usepackage[left=3cm,right=2.5cm,top=3.0cm,bottom=3.0cm]{geometry} %Ränder %%%% PRINT
\usepackage[left=2.75cm,right=2.75cm,top=3cm,bottom=3cm]{geometry} %Ränder 
\usepackage[nottoc]{tocbibind}

%schoenere Kopf- und Fusszeilen, für ein und doppelseitig
\usepackage{fancyhdr} \pagestyle{fancy}
\renewcommand\headrulewidth{0.4pt}
\fancyhead[ER,OL]{} 
\fancyfoot[C]{\thepage}
\fancyhead[EL,OR]{\textsl{\leftmark}} % nur die section oben links (nicht subsect)

% eine seite ohne sectionsnamen oben sondern nur mit linie
\fancypagestyle{justline}{
\fancyhead[R]{}
\renewcommand\headrulewidth{0.4pt}
\fancyfoot[C]{\thepage}
\fancyhead[OL,ER]{}
\fancyhead[EL,OR]{} }

\usepackage{footnpag} %fussnoten werden auf jeder Seite einzeln gezaehlt

\addto{\captionsngerman}{%
  %\renewcommand*{\contentsname}{Inhalt}
  \renewcommand*{\listfigurename}{Abbildungsverzeichnis}
  \renewcommand*{\listtablename}{Tabellenverzeichnis}
  \renewcommand*{\figurename}{Abb.}
  \renewcommand*{\tablename}{Tab.}
}
\usepackage{hyperref} % Links im PDF
\usepackage{lineno} %Zeilennummern von \linenumbers bis \nolinenumbers

\usepackage{anyfontsize}
\usepackage[textsize=scriptsize]{todonotes} % todos
\setlength{\marginparwidth}{2cm} % wie weit die todos nach rechts dürfen
\title{Verfassen von Hausarbeiten mit LaTeX}
\author{}

\usepackage{wrapfig}

%-----------Head Ende------------------------------------
\begin{document}
\renewcommand{\arraystretch}{1.2} % entwas entzerrtere Tabellen
\begin{titlepage}

\begin{center}
  \bfseries
  \LARGE Universität Osnabrück
  \vskip.6in
  \Large Studienprojekt %Worum handelt es sich? Hausarbeit/ Essay/ Aufgabenblatt
  \vskip0.15in 
  \small Betreut durch
  %\vskip.15in
  \Large %Studienprojekt  %Welches Modul?
  \vskip.3in
  \small  Prof. Dr. Frank Hilker
  \vskip.3in
  \small Wintersemester 2023 %Welches Semester
  \vskip.5in
  \vfill
  \rule{\textwidth}{0.25mm}
  \vskip0.25cm
  \LARGE Host-Parasitoid Interactions
  \vskip0.cm
  \rule{\textwidth}{0.25mm}
  \vskip1in
  
  \vfill
 \small verfasst von 
 \vskip.5in
\end{center}


\begin{minipage}{1\textwidth}
\begin{center}
\bfseries\large 
Robin  \\ Elias \\
\vskip.1in
968504\\ % Matrikelnummer
\vskip.1in
\small \textit{Umweltsystemwissenschaft Monobachelor} \\

\end{center}
\end{minipage}
\hfill


\vfill
\begin{center}
    Abgegeben am \today
\end{center}
\end{titlepage}

\thispagestyle{justline}
\setcounter{page}{1} % Seite eins ist hier
\setcounter{tocdepth}{2} % tocdepth 2 enthält bis zu subsection, 3 bis zu subsubsections
\tableofcontents
\listoffigures
\newpage

%%%%%%%%%%%%%%%%%
%Ab hier beginnt das Dokument
%%%%%%%%%%%%%%%%%

\section*{A}

The project was implemented in \href{https://docs.julialang.org/en/v1/}{Julia} and uses the DrWatson package.
To reproduce the results look at README.md on \href{https://github.com/relias96/Studienprojekt}{github.com}

\section{Introduction}

The development of models to describe host-parasitoid interactions began as early as the beginning of the 20th century.
Motivated by the possibility of regulating pests in agriculture by the controlled application of 
parasitoids, various approaches were developed. Today, as then, research faces a number of challenges. 
Although the data has grown over the years and numerous models have been developed, there is still a 
lack of mechanistic understanding of the interactions and complex dynamics that can occur in such a system.
 
Since the models can exhibit highly non-linear behavior, analytical mathematic quickly reaches its limits. 
Thanks to the rapid development of computer technology computers are now available to researchers. 
In recent years, therefore numerical algorithms have been developed to investigate complex system behavior.


\section{Model and Methods}

\subsection{The Model}
The modeling of host-parasitoid interactions has a long history. Since insects often have separate generations, 
discrete-time models are suitable. In general all our models base on the following assumptions:

\begin{equation}
\begin{split}
  N_{t+1} &= g(N_t)  f(N_t,P_t) \\
  P_{t+1} &= c N_t[1- f(N_t, P_t)] \\
\end{split}
\end{equation}

Given this equations $g(N_t)$ describes the growth of the hosts, $f(N_t, P_t)$ the proportion of hosts
that is not infested by parasites and $c$ indicates the average number of parasites that emerge from 
a parasitized host.

Previos research often build on the assumption that the growth of host following the Ricker map given
by:
\begin{equation}
  g(N_t)= N_t e^{r(1-N_t)}
\end{equation}

About this Ricker Model is known that in respect to the intrinsic growth rate $r$ the behavior changes 
from stable, over period doubling to chaos.  However we want to reduce this impact, therefore we use 
the Beverton-Holt map as a also well known model. This model is also a discrete-time equivalent to 
logistic growth, which shows stable behavior and approaches the carring capacity in every permitted parameter set. 

The Beverton-Holt Model ist given by the following equation
\begin{equation}
  \begin{split}
    g(N_t) = \frac{\lambda N_t}{1 + \frac{(\lambda -1)N_t}{K}}
  \end{split}
\end{equation}

with $\lambda > 0$ as the intrinsic growth rate and $K > 0$ as the capacity.\\

Assuming that the interaction between host and parasitoid is density-dependent with a Holling type
\Romannum{3} results:

\begin{equation}
  \begin{split}
    f(N_t, P_t) = 1 - e^{- \frac{N_{enc}}{N_t}} =1- e^{\left[- \frac{aTN_t P_t}{1+cN+aT_h N_t^2} \right]} \\
  \end{split}
\end{equation}

with the total time per generation $T$, the handling time $T_h$ and the parasitization rate $a$.\\

Through dedimensionalization with
\begin{equation}
  n_t = \frac{N_t}{K} \hspace*{10mm} p_t=\frac{P_t}{bK} \hspace*{10mm} \alpha = abTK^2 \hspace*{10mm} \mu = aT_h K^2
\end{equation}
one receives:
\begin{equation}
  \begin{split}
    n_{t+1} &= \frac{\lambda n_t}{1+(\lambda -1) n_t} e^{\frac{-\alpha n_t p_t}{1-\mu n_t^2}}\\
    p_{t+1} &= n_t \left( 1-e^{\frac{-\alpha n_t p_t}{1+\mu n_t^2}}\right)
  \end{split}
\end{equation}
with the scaled growth rate $\lambda > 0$, the scaled parasitization rate $\alpha > 0$ and 
the scaled handling time $\mu > 0$

\subsection{Analysemethoden}
The model shows the following complex behavior:
\begin{itemize}
  \item Non-unique dynamics, which mean coexisting attractors
  \item Basin of attraction
  \item Intermittency
  \item Supertransient behavior
\end{itemize}



\subsubsection{Implementations}

Because of the complex behavior the existance and stability of an Attractor is hardly 
analytical shown. Therefore we investigate this behavior by using a few numerical methods. 
We choose the Julia programming Language because many required functionalities are already implemented, 
with easy readable code and great performance.\par

For numerically evolving the given System we use the \texttt{trajectory} function from the \texttt{DynamicalSystems} 
libary in the Julia programming language.\par

For Attractor detection we use the \texttt{AttractorsViaRecurrences} function from the \texttt{Attractors} 
libary by  Datseries and Wagemakers which relies on a method described by Nusse and Yorke.\par

Given this set of Attractors one can match the inital conditions with one of the Attractors to recieve the 
basin of attraction. This  is also implemented in the \texttt{Attractors} libary with the function 
\texttt{basins_of_attraction}.

\subsubsection{Example}
To get the basin of attraction we use a finate state machine. The state space is devided into a discrete 
grid initialized with a value \texttt{u} which means \textit{unvisited}\par

For each inital condition we track the trajectory of the System. Once an Attractor hits a cell the value 
changes to \texttt{v} for \textit{visited}. When the trajectory hits consecutive times previos visited cells 
an Attractor is detected and the inital condition is attached to the Attractor.\par

For a more detailed description see `Effortless estimation of basin of attraction' by Datseries.

\section{Ergebnisse}


\section{Diskursion}

The given System is a fully deterministic Model. Therefore fractal properties do not directly influence 
the stability of an Attractor. But under respect of random fluctuations, mixed up basins with close proximaty 
of different Attractors may disturb the System. A metric on the basin based on the neighborhood could quantify this
insecurity. 



\newpage
\thispagestyle{justline}
{\small
\bibliographystyle{apalike-german}
%\bibliography{xyz} %hier den Namen der .bib Datei 
}


\end{document}