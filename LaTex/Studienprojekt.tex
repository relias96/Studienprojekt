

\documentclass[a4paper,12pt, twoside]{article} % twoside für doppelseitigen ausdruck

\usepackage[utf8]{inputenc} % für Umlate
\usepackage[english]{babel} % deutsches TOC

\usepackage[autostyle]{csquotes}

\usepackage[
    backend=biber,
    style=authoryear-icomp,
    sortlocale=de_DE,
    natbib=true,
    url=false, 
    doi=true,
    eprint=false
]{biblatex}
\addbibresource{Studienprojekt.bib}

\usepackage[]{hyperref}
\hypersetup{
    colorlinks=false,
}

%\usepackage[babel]{csquotes} 
%\usepackage[authoryear]{natbib} % Zitate  Autor (Jahr), bzw. (Autor, Jahr)

\usepackage{url} % Für Links im Text
\usepackage{float} % H als feste Position
\usepackage[font=footnotesize,labelfont=bf]{caption} % für schöner formatierte Captions
\usepackage[shortlabels]{enumitem} % \begin{enumerate}[a)] für a) enumerate
\usepackage{graphicx} % für bessere Grafiken
\usepackage[singlespacing]{setspace} % 1,5 facher Zeilenabstand
\usepackage{makecell}

\usepackage{booktabs}
\usepackage{multirow}
\usepackage{underscore}
\usepackage{amsmath}
\usepackage{array}
\usepackage{romannum}
\usepackage{parskip}
\usepackage{pdflscape}

\usepackage{algorithm}
\usepackage{algpseudocode}


\makeatletter
\def\BState{\State\hskip-\ALG@thistlm}
\makeatother

%Formatierung von Zitaten
%Doppelpunkt vor der Seitenzahl
%Zitate in () Klammern
%mehrere Zitate mit ; trennen
%a Autor und Jahr
%, Trennzeichen zwischen Autor und Jahr
%,~ Trennzeichen bei mehreren Jahresangaben
%\bibpunct[:~]{(}{)}{;}{a}{,}{,~}

\setlength\parindent{0pt} % kein linker Einzug bei neuen pars
%\usepackage[left=3cm,right=2.5cm,top=3.0cm,bottom=3.0cm]{geometry} %Ränder %%%% PRINT
\usepackage[left=2.75cm,right=2.75cm,top=3cm,bottom=3cm]{geometry} %Ränder 
\usepackage[nottoc]{tocbibind}

%schoenere Kopf- und Fusszeilen, für ein und doppelseitig
\usepackage{fancyhdr} \pagestyle{fancy}
\renewcommand\headrulewidth{0.4pt}
\fancyhead[ER,OL]{} 
\setlength{\headheight}{14.49998pt}
\fancyfoot[C]{\thepage}
\fancyhead[EL,OR]{\textsl{\leftmark}} % nur die section oben links (nicht subsect)

% eine seite ohne sectionsnamen oben sondern nur mit linie
\fancypagestyle{justline}{
\fancyhead[R]{}
\renewcommand\headrulewidth{0.4pt}
\fancyfoot[C]{\thepage}
\fancyhead[OL,ER]{}
\fancyhead[EL,OR]{} }

\usepackage{footnpag} %fussnoten werden auf jeder Seite einzeln gezaehlt

\addto{\captionsngerman}{%
  %\renewcommand*{\contentsname}{Inhalt}
  \renewcommand*{\listfigurename}{Abbildungsverzeichnis}
  \renewcommand*{\listtablename}{Tabellenverzeichnis}
  \renewcommand*{\figurename}{Abb.}
  \renewcommand*{\tablename}{Tab.}
}
%\usepackage{hyperref} % Links im PDF
%\usepackage{lineno} %Zeilennummern von \linenumbers bis \nolinenumbers

\usepackage{anyfontsize}
\usepackage[textsize=scriptsize]{todonotes} % todos
\setlength{\marginparwidth}{2cm} % wie weit die todos nach rechts dürfen

\usepackage{wrapfig}

%-----------Head Ende------------------------------------
\begin{document}
\renewcommand{\arraystretch}{1.2} % entwas entzerrtere Tabellen
\begin{titlepage}

\begin{center}
  \bfseries
  \LARGE Universität Osnabrück
  \vskip.6in
  \Large Studienprojekt %Worum handelt es sich? Hausarbeit/ Essay/ Aufgabenblatt
  \vskip0.15in 
  \small Betreut durch
  %\vskip.15in
  \Large %Studienprojekt  %Welches Modul?
  \vskip.3in
  \small  Prof.\ Dr.\ Frank Hilker
  \vskip.3in
  \small Wintersemester 2023 %Welches Semester
  \vskip.5in
  \vfill
  \rule{\textwidth}{0.25mm}
  \vskip0.25cm
  \LARGE Host-Parasitoid Interactions
  \vskip0.cm
  \rule{\textwidth}{0.25mm}
  \vskip1in
  
  \vfill
 \small verfasst von 
 \vskip.5in
\end{center}


\begin{minipage}{1\textwidth}
\begin{center}
\bfseries\large 
Robin  \\ Elias \\
\vskip.1in
968504\\ % Matrikelnummer
\vskip.1in
\small \textit{Bachelor of Science} \\

\end{center}
\end{minipage}
\hfill


\vfill
\begin{center}
    Abgegeben am \today
\end{center}
\end{titlepage}

\thispagestyle{justline}
\setcounter{page}{1} % Seite eins ist hier
\setcounter{tocdepth}{2} % tocdepth 2 enthält bis zu subsection, 3 bis zu subsubsections
\tableofcontents
\listoffigures
\newpage

%%%%%%%%%%%%%%%%%
%Ab hier beginnt das Dokument
%%%%%%%%%%%%%%%%%



\section{Introduction}
The Earth's biodiversity is made up of around 1.7 million recorded species. Of these, about 57\% are insects, 25\% plants, 
fungi and microorganisms and 20\% vertebrates and other animals. Therefore there are nearly 1 million different insect species. \autocite[S. 35ff]{biodiversity}
About 10\% of all described insect species are parasitoids \autocite{eggleton1997}, 
which corresponds to about 100 thousand species. This leads to the conclusion that parasitism is a realy common thing to expect in a natural
environment. For this reason Thompson developed the first host-parasitoid models in the beginning of the 20th century, motivated by the possibility 
of regulating agricultural pests through the controlled use of parasitoids. These models were very simple and their predictions rather divorced
from the often complex behavior found in the field. Therefore numerous other models have been developed. But as today as then, research faces a number 
of challenges. Although the data has grown over the years there is still a lack of mechanistic understanding of the interactions 
and complex dynamics that can occur in such a real life system or corresponding model. 
 
When such highly non-linear behavior occurs, analytical mathematic's quickly reaches it's limits. 
However, thanks to the rapid development of computer technology and numerical algorithms for investigating such complex system behavior, it is now 
possible to gain insights into such models. This report is intended to show how efficient and convincing these algorithms have become 
and what results can be expected and achieved. For this purpose, we use cutting-edge algorithms developed in the Julia programming language.


\section{Model and Methods}

\subsection{The Model}
The modeling of host-parasitoid interactions has a long history. Since insects often have separate generations, 
discrete-time models are suitable. In general all our models base on the following assumptions:

\begin{equation}
\begin{split}
  N_{t+1} &= g(N_t)  f(N_t,P_t) \\
  P_{t+1} &= b N_t[1- f(N_t, P_t)] \\
\end{split}
\end{equation}

Given this equations $g(N_t)$ describes the growth of the hosts. Because parasitoid act lethal on the host population, only not infested one are fertile and can reproduce. Therefore $f(N_t, P_t)$ is the proportion of hosts
that is not infested by parasites. The parameter $b$ indicates the average number of parasites that emerge from 
one parasitized host.

Previos research by \autocite[][]{Kaitala} build on the assumption that the growth of host following the Ricker map given
by:
\begin{equation}
  g(N_t)= N_t e^{r(1-N_t)}
\end{equation}

About this Ricker map is known that in respect to the intrinsic growth rate $r$ the behavior changes 
from stable, over period doubling to chaos.  However we want to reduce this impact, therefore we use 
the Beverton-Holt map. This map is also a discrete-time equivalent to 
logistic growth, which shows stable behavior and approaches the carring capacity in every permitted parameter set. 

The Beverton-Holt Model ist given by the following equation
\begin{equation}
  \begin{split}
    g(N_t) = \frac{\lambda N_t}{1 + \frac{(\lambda -1)N_t}{K}}
  \end{split}
\end{equation}

with the intrinsic growth rate $\lambda > 0$  and the capacity $K > 0$.

We assume that the interaction between host and parasitoid is density-dependent with a holling type
\Romannum{3} and that the host and the parasitoid encounter randomly $N_{enc}$ times. 
Since it is possible that one host can be encountert multiple times by different parasitoids.
 The number each single host is encountered is poisson distributed and
 therefore the zero-term represents the proportion of not infected host.
 
 This results in:

\begin{equation}
  \begin{split}
    f(N_t, P_t) = e^{- \frac{N_{enc}}{N_t}} =e^{\left[- \frac{aTN_t P_t}{1+cN_t+aT_h N_t^2} \right]} 
  \end{split}
\end{equation}

with the per capita encounters $\frac{N_{enc}}{N_t}$,  the total time per generation $T$, the handling time $T_h$ and the parasitization rate $a$. \autocite[see. chapter 2.4.1]{hassell}

Through dedimensionalization with
\begin{equation}
  n_t = \frac{N_t}{K} \hspace*{10mm} p_t=\frac{P_t}{bK} \hspace*{10mm} \alpha = abTK^2 \hspace*{10mm} \mu = aT_h K^2
\end{equation}
one receives:
\begin{equation}
  \begin{split}
    n_{t+1} &= \frac{\lambda n_t}{1+(\lambda -1) n_t} e^{\frac{-\alpha n_t p_t}{1+\mu n_t^2}}\\
    p_{t+1} &= n_t \left( 1-e^{\frac{-\alpha n_t p_t}{1+\mu n_t^2}}\right)
  \end{split}
\end{equation}
with the growth rate $\lambda > 0$, the scaled parasitization rate $\alpha > 0$ and 
the scaled handling time $\mu > 0$



\subsection{The Algorithms}
The project was implemented in \href{https://docs.julialang.org/en/v1/}{Julia}\footnote[1]{https://docs.julialang.org/en/v1/} and uses the DrWatson package.
To reproduce the results look at README.md on \href{https://github.com/relias96/Studienprojekt}{github.com}\footnote[2]{https://github.com/relias96/Studienprojekt}

Because of the complex behavior the existance and stability of an Attractor is hardly 
analytical shown. Therefore we investigate this behavior by using a few numerical methods. 
We choose the Julia programming Language because many required functionalities are already implemented, 
with easy readable code and great performance.\par

\subsubsection{Orbit-diagramm}
An orbit-diagramm is often mistakenly called bifurcation-diagramm. One thing in common is that a parameter is varied over a range, but while an orbit-diagramm only 
shows attracting equilibria/cycles or possible locations of chaos, a bifurcation diagram in contrast plots all attracting and repelling periodic points \autocite{bifurcation}.

While the computation of the bifurcation diagram can be very complex and time-consuming, the computation of an orbit diagram is relatively simple. Since an orbit diagramm 
is sufficient for our research, we will use it in the following. Because there are co-existing attractors, we implement a algorithm in the following manner:


For each parameter value on the x-axis, we randomly vary the initial conditions several times to depict even unlikely attractors. Then we evolve the system \texttt{m} times and record the following \texttt{n} states. If the attractor is chaotic,
which means that the number of attractor-states with unique values is greater than the limit up to which we want to capture periodic behavior, we plot blurred dots. This ensures that 
the periodic and stable points are in contrast with the chaotic ones. 

\begin{algorithm}[H]
  \caption{Orbitdiagram}\label{orbit}
  \begin{algorithmic}[1]
  \Function{timeseries}{system, n, m}
    \State evolve system m times
    \State $\textit{ts} \gets \text{evolve system } \textit{n} \text{ times}$
    \State \Return \textit{ts}
  \EndFunction
  \Procedure{PlotOrbit}{}
    \State $\texttt{system} \gets \texttt{Dict}\text{(equationOfMotion; parameter; initialState)}$
    \State $\texttt{m} \gets \text{transient time}$
    \State $\texttt{n} \gets \text{tracked steps}$
    \State $\texttt{p\_values} \gets \texttt{range}(parameter)$
    \State $\texttt{limit} \gets \text{maximal period}$
    \ForAll{$\textit{p} \in \text{parameter\_values}$}
      \Repeat
        \State $\textit{system.initialState} \gets \textit{u_0}$
        \State $attractor \gets Set(\Call{timeseries}{system, n, m})$
        \If{$length(attractor) < limit$}
          \State $\text{\Call{Plot}{p,attractor} as dots}$
        \Else 
          \State $\text{\Call{Plot}{p,attractor} as blurred dots}$
        \EndIf
      \Until{Plot is detailed enough}
    \EndFor
  \EndProcedure
  \end{algorithmic}
\end{algorithm}




\subsubsection{Basin of attraction}

If there are at least two coexisting attractors that depend on the initial condition, the basin of attraction proves to be a valuable tool. 
It uses a numerical approach to partition the statespace of initial conditions into different regions. 
Each of these regions represents a set of initial conditions that lead to its corresponding attractor.

\begin{figure}[H]
  \begin{centering}
    \includegraphics[width =.9\textwidth]{../plots/basinOfAttraction.png} 
    \caption{Basin of Attraction}\label{fig:basinAlgo}
  \end{centering}
\end{figure}

Figure \ref{fig:basinAlgo} shows the algorithm behind the function \texttt{basins_of_attraction} which relies on a method described in Chapter 7 of \autocite{nusse2012dynamics}.
To get the basin of attraction we use a finite state machine. The state space is devided into a discrete grid initialized with a value \textit{unvisited}.
For each inital condition we track the trajectory of the System by using the mapper \texttt{AttractorsViaRecurrences} which numerically evolves the given System using the \texttt{trajectory} function 
from the \texttt{DynamicalSystems} libary and tracks its path in the Statespace. Once the trajectory hits a \textit{unvisited} cell the value changes to  \textit{visited} (upper left of Figure \ref{fig:basinAlgo}).

The Algorithm terminates when one counter hits its following corresponding threshold:
\begin{itemize}[itemsep=-4pt]
  \item \texttt{consecutive_recurrences} times an \textit{visited} cells get hit
  \item \texttt{consecutive_basin_steps} times basin cells with same \textit{ID} gets hit
  \item \texttt{consecutive_attractor_steps} times attractor cells with same \textit{ID} get hit
\end{itemize}

When the \texttt{consecutive_recurrences} threshold is reached the new attractor is found and the cell in the grid coresponding to the initial condition gets an new \textit{ID} which identifies the the basin to the attractor.
Otherwise when \texttt{consecutive_basin_steps} or \texttt{consecutive_attractor_steps} threshold is reached the existing \textit{ID} gets assigned to the cell, adding it to an existing basin. 

Following this algorithm each inital conditions can be matched with one of the Attractors to receive the basin of attraction.



\begin{algorithm}[H]
  \caption{Algorithm to compute the basin of attraction}\label{BoA}
  \begin{algorithmic}
  \Procedure{basinOfAttraction}{}
    \State $\texttt{system} \gets \texttt{Dict}\text{(equationOfMotion; parameter; initialState)}$
    \State $\texttt{grid} \gets \text{grid of empty cells} $
    \State $\texttt{attractors} \gets \text{[ ]}$
    \ForAll{$\text{cell} \in \texttt{grid}$}
      \State $\texttt{coordinate, ID} \gets \text{cell}$
      \State $\texttt{system}\text{.initialState} \gets \texttt{coordinate}$
      \State $\texttt{ID} \gets \Call{AttractorsViaRecurrences}{\texttt{system, grid, attractors}}$
    \EndFor
  \EndProcedure
\end{algorithmic}
\end{algorithm}

\begin{algorithm}[H]
  \caption{Algorithm mapping an initial State to an Attractor}\label{AvR}
  \begin{algorithmic}
  \Function{AttractorsViaRecurrences}{system, grid, attractors}
    \State $\texttt{tresholds}$
    \State $\texttt{counters}$
    \While{$\forall \texttt{ counters} < \texttt{tresholds} $}
      \State $\texttt{next_cell} \gets \Call{evolve}{system, \Delta t } $
      \State $\Call{adjustCounters}{\texttt{counters, next_cell}}$
      \If {$\texttt{next_cell} == empty$}
      \State $\texttt{next_cell} = visited$
      \EndIf
    \EndWhile
    \If {\texttt{consecutive_recurrences} \text{ treshold is reached}}
    \State $\text{new_attractor} \gets  \Call{GetAttractor}{system, grid}$
    \State $\text{attractors}.\Call{add}{\text{new_attractor}}$
    \State \Return $\text{newID}$
    \EndIf
    \If {$\texttt{consecutive_basin_steps} \text{ treshold is reached}$}
      \State $\text{ID} \gets \text{BasinID}$
      \State \Return $\text{ID}$
    \EndIf
    \If {$\texttt{consecutive_attractor_steps} \text{ treshold is reached}$}
      \State $\text{ID} \gets \text{AttractorID}$
      \State \Return $\text{ID}$
    \EndIf
  \EndFunction
\end{algorithmic}
\end{algorithm}

For a more detailed description see `Effortless estimation of basin of attraction' by Datseries \autocite{datseris} or have a look into the \href{https://juliadynamics.github.io/Attractors.jl/dev/attractors/} {documentation}\footnote[1]{https://juliadynamics.github.io/Attractors.jl/dev/attractors/} .
\section{Results}
Since we are using dedimensionalised equations and do not have a real-life scenario to fit them into, our aim is to gain a qualitative understanding of the model.
Previous research demonstrated the peculiar behaviour of their Host-Parasitoid Model when assuming host growth following a Ricker map \autocite{Kaitala}.
To mitigate the destabilizing impacts caused by the chaotic behaviour of the Ricker map at certain parameter values, the Beverton-Holt map was selected for its stable behaviour across all parameters.




\subsection{Transient behavior}
At first we discovered long transient phases before the Model reaches its final state. This means that it takes a unexpected long time of inconclusive behavior before the behavior settles
to a state. In Figure \ref{fig:transient} the Model ($\lambda=2.5$; $\alpha = 17.23 $; $ \mu = 0.1$) reaches it's Equilibrium after 8000 generations. 

\begin{figure}[H]
\begin{centering}
\includegraphics[width =\textwidth]{../plots/Project_Report/transient.png}
\caption{Transient behavior}\label{fig:transient}
\end{centering}
\end{figure}

\subsection{Intermittency}

For some parameter combination intermittency occurs in our Model. 
This means that a equlibrium is no longer stable, but instead switches randomly between states. 
In figure \ref{fig:intermittency} the timeseries for the parameters $\lambda=6.5$; $\alpha = 21.411 $; $ \mu = 0.1$ is plotted.
It is rare to find in this model, but it is possible. 
\begin{figure}[H]
\begin{centering}
\includegraphics[width =\textwidth]{../plots/Project_Report/Intermittency.png}
\caption{Intermittency}\label{fig:intermittency}
\end{centering}
\end{figure}

\subsection{Non-unique dynamics and basin of attraction}

Our Model can include several coexisting attractors. This means that for the same parameters the system can end up in different attractors. To explore the presence of the attractors we 
make use of the orbitdiagram defined in Algorithm \ref{orbit}, to track all attractors for a given parameter combination. 

For the Parameters $\lambda = 7$ and $\mu = 2.5$ the following plots are produced.

\begin{figure}[H]
  \begin{centering}
    \includegraphics[width =.9\textwidth]{../plots/Project_Report/Orbit14-23.png} 
    \caption{Orbitdiagram}
    \label{fig:Orbitdiagram}
  \end{centering}
\end{figure}

\begin{figure}[H]
  \begin{centering}
    \includegraphics[width =0.8\textwidth]{../plots/Project_Report/orbit Zoom.png} 
    \caption{Detailed-Orbitdiagram}
    \label{fig:Orbitdiagram2}
  \end{centering}
\end{figure}

When $\alpha < 14$ there is only on stable attractor. At around 14 the stable attractor changes to an quasiperiodic attractor with an growing period.
At around $\alpha = 15.5$ a periodic attractor with period 5 appears, which persists for higher $\alpha$-values.

At around $\alpha = 20$ chaotic behaviour can appear. Multiple times an additional attractor arises which first
 goes through period doubling and then turns into a chaotic attractor or disappears. 
 Whenever there are multiple coexisting attractors the algorithm defined in Algorithm \ref{BoA} can be used to get the boundaries in the statespace to distinguish between the attractors.
 In Figure \ref{fig:basin1} and \ref{fig:basin2} two basins of attraction for $\lambda = 7$ and $\mu = 2.5$

Figure \ref{fig:basin1} shows a basin of attraction for $\alpha = 21.5$. The Basin on the left hand side contains three attractors, one chaotic(black) and two periodic(purple and turquoise).
The turquoise one has fractal properties and the chaotic one appears as closed curve in the Statespace with its typical stretching and folding properties. 
In Figure \ref{fig:basin2} the chaotic attractor dissapears and instead two stable riddled attractors emerge. Therefore there are four coexisting periodic attractors. 

 \begin{landscape}
\begin{figure}[H]
  \begin{centering}
    \includegraphics[width =1.5\textwidth]{../plots/Project_Report/Basins/215.png}
    \caption{Basin of attraction: $\alpha = 21.5$}
    \label{fig:basin1}
  \end{centering}
\end{figure}


\begin{figure}[H]
  \begin{centering}
    \includegraphics[width =1.5\textwidth]{../plots/Project_Report/Basins/216.png}
    \caption{Basin of attraction: $\alpha = 21.6$}
    \label{fig:basin2}
  \end{centering}
\end{figure}
\end{landscape}
The results have shown that even when using the Beverton-Holt map, the behavior can be very complicated and chaotic.

\section{Diskursion}
Mathematically, our results show the presence of several coexisting attractors. Coexisting in the sense of arising from the same combination of parameters, but differing from each other in the initial conditions.


In reality, however, a population cannot have two different initial conditions and therefore only one attractor can be achieved. 
Since the model is deterministic, each of the possible attractors is stable but sensitive to population fluctuations. This means that our system is multi-stable.
Thus, random changes or external influences such as immigration or emigration can change the system behaviour by switching between attractors. 


It can be assumed, that the attractors have different sensitivities to fluctuations in phase space. 
Since the basin plot assigns an attractor to each state, and the attractor is also only a set of states,
a metric can be derived from this, which determines the sensitivity of an attractor based on the neighborhood relationships.


Overall the Algorithms provided by the \texttt{DynamicalSystems} libary proofed to be very usefull and efficient. Without these it would not be possible to investigate the more complex behavior. The
Julia programming language ist easy to understand and well documented and therefore not only fast in computation, but also fast to implement. The libary
contains many more usefull functionalities like in the ChaosTools\footnote[1]{https://juliadynamics.github.io/ChaosTools.jl/dev/} module. 

\thispagestyle{justline}
\printbibliography[]


\end{document}