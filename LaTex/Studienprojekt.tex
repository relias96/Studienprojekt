%!LW recipe=pdflatex

\documentclass[a4paper,12pt, twoside]{article} % twoside für doppelseitigen ausdruck

\usepackage[utf8]{inputenc} % für Umlate
\usepackage[english]{babel} % deutsches TOC

\usepackage[autostyle]{csquotes}

\usepackage[
    backend=biber,
    style=authoryear-icomp,
    sortlocale=de_DE,
    natbib=true,
    url=false, 
    doi=true,
    eprint=false
]{biblatex}
\addbibresource{Studienprojekt.bib}

\usepackage[]{hyperref}
\hypersetup{
    colorlinks=false,
}

%\usepackage[babel]{csquotes} 
%\usepackage[authoryear]{natbib} % Zitate  Autor (Jahr), bzw. (Autor, Jahr)

\usepackage{url} % Für Links im Text
\usepackage{float} % H als feste Position
\usepackage[font=footnotesize,labelfont=bf]{caption} % für schöner formatierte Captions
\usepackage[shortlabels]{enumitem} % \begin{enumerate}[a)] für a) enumerate
\usepackage{graphicx} % für bessere Grafiken
\usepackage[singlespacing]{setspace} % 1,5 facher Zeilenabstand
\usepackage{makecell}

\usepackage{booktabs}
\usepackage{multirow}
\usepackage{underscore}
\usepackage{amsmath}
\usepackage{array}
\usepackage{romannum}
\usepackage{parskip}

\usepackage{algorithm}
\usepackage[noend]{algpseudocode}

\makeatletter
\def\BState{\State\hskip-\ALG@thistlm}
\makeatother

%Formatierung von Zitaten
%Doppelpunkt vor der Seitenzahl
%Zitate in () Klammern
%mehrere Zitate mit ; trennen
%a Autor und Jahr
%, Trennzeichen zwischen Autor und Jahr
%,~ Trennzeichen bei mehreren Jahresangaben
%\bibpunct[:~]{(}{)}{;}{a}{,}{,~}

\setlength\parindent{0pt} % kein linker Einzug bei neuen pars
%\usepackage[left=3cm,right=2.5cm,top=3.0cm,bottom=3.0cm]{geometry} %Ränder %%%% PRINT
\usepackage[left=2.75cm,right=2.75cm,top=3cm,bottom=3cm]{geometry} %Ränder 
\usepackage[nottoc]{tocbibind}

%schoenere Kopf- und Fusszeilen, für ein und doppelseitig
\usepackage{fancyhdr} \pagestyle{fancy}
\renewcommand\headrulewidth{0.4pt}
\fancyhead[ER,OL]{} 
\setlength{\headheight}{14.49998pt}
\fancyfoot[C]{\thepage}
\fancyhead[EL,OR]{\textsl{\leftmark}} % nur die section oben links (nicht subsect)

% eine seite ohne sectionsnamen oben sondern nur mit linie
\fancypagestyle{justline}{
\fancyhead[R]{}
\renewcommand\headrulewidth{0.4pt}
\fancyfoot[C]{\thepage}
\fancyhead[OL,ER]{}
\fancyhead[EL,OR]{} }

\usepackage{footnpag} %fussnoten werden auf jeder Seite einzeln gezaehlt

\addto{\captionsngerman}{%
  %\renewcommand*{\contentsname}{Inhalt}
  \renewcommand*{\listfigurename}{Abbildungsverzeichnis}
  \renewcommand*{\listtablename}{Tabellenverzeichnis}
  \renewcommand*{\figurename}{Abb.}
  \renewcommand*{\tablename}{Tab.}
}
%\usepackage{hyperref} % Links im PDF
%\usepackage{lineno} %Zeilennummern von \linenumbers bis \nolinenumbers

\usepackage{anyfontsize}
\usepackage[textsize=scriptsize]{todonotes} % todos
\setlength{\marginparwidth}{2cm} % wie weit die todos nach rechts dürfen

\usepackage{wrapfig}

%-----------Head Ende------------------------------------
\begin{document}
\renewcommand{\arraystretch}{1.2} % entwas entzerrtere Tabellen
\begin{titlepage}

\begin{center}
  \bfseries
  \LARGE Universität Osnabrück
  \vskip.6in
  \Large Studienprojekt %Worum handelt es sich? Hausarbeit/ Essay/ Aufgabenblatt
  \vskip0.15in 
  \small Betreut durch
  %\vskip.15in
  \Large %Studienprojekt  %Welches Modul?
  \vskip.3in
  \small  Prof.\ Dr.\ Frank Hilker
  \vskip.3in
  \small Wintersemester 2023 %Welches Semester
  \vskip.5in
  \vfill
  \rule{\textwidth}{0.25mm}
  \vskip0.25cm
  \LARGE Host-Parasitoid Interactions
  \vskip0.cm
  \rule{\textwidth}{0.25mm}
  \vskip1in
  
  \vfill
 \small verfasst von 
 \vskip.5in
\end{center}


\begin{minipage}{1\textwidth}
\begin{center}
\bfseries\large 
Robin  \\ Elias \\
\vskip.1in
968504\\ % Matrikelnummer
\vskip.1in
\small \textit{B.Sc.} \\

\end{center}
\end{minipage}
\hfill


\vfill
\begin{center}
    Abgegeben am \today
\end{center}
\end{titlepage}

\thispagestyle{justline}
\setcounter{page}{1} % Seite eins ist hier
\setcounter{tocdepth}{2} % tocdepth 2 enthält bis zu subsection, 3 bis zu subsubsections
\tableofcontents
\listoffigures
\newpage

%%%%%%%%%%%%%%%%%
%Ab hier beginnt das Dokument
%%%%%%%%%%%%%%%%%



\section{Introduction}
The Earth's biodiversity is made up of around 1.7 million recorded species. Of these, about 57\% are insects, 25\% plants, 
fungi and microorganisms and 20\% vertebrates and other animals. Therefore there are nearly 1 million different insect species. \autocite[S. 35ff]{biodiversity}
About 10\% of all described insect species are parasitoids \autocite{eggleton1997}, 
which corresponds to about 100 thousand species. This leads to the conclusion that parasitism is a realy common thing to expect in a natural
environment. For this reason Thompson developed the first host-parasitoid models in the beginning of the 20th century, motivated by the possibility 
of regulating agricultural pests through the controlled use of parasitoids. These models were very simple and their predictions rather divorced
from the often complex behavior found in the field. Therefore numerous other models have been developed. But as today as then, research faces a number 
of challenges. Although the data has grown over the years there is still a lack of mechanistic understanding of the interactions 
and complex dynamics that can occur in such a real life system or corresponding model. 
 
When such highly non-linear behavior accurs, analytical mathematic's quickly reaches it's limits. 
However, thanks to the rapid development of computer technology and numerical algorithms for investigating such complex system behavior, it is now 
possible to gain insights into such models. This report is intended to show how efficient and convincing these algorithms have become 
and what results can be expected and achieved. For this purpose, we use cutting-edge algorithms developed in the Julia programming language.


\section{Model and Methods}

\subsection{The Model}
The modeling of host-parasitoid interactions has a long history. Since insects often have separate generations, 
discrete-time models are suitable. In general all our models base on the following assumptions:

\begin{equation}
\begin{split}
  N_{t+1} &= g(N_t)  f(N_t,P_t) \\
  P_{t+1} &= c N_t[1- f(N_t, P_t)] \\
\end{split}
\end{equation}

Given this equations $g(N_t)$ describes the growth of the hosts, $f(N_t, P_t)$ the proportion of hosts
that is not infested by parasites and $c$ indicates the average number of parasites that emerge from 
a parasitized host.

Previos research often build on the assumption that the growth of host following the Ricker map given
by:
\begin{equation}
  g(N_t)= N_t e^{r(1-N_t)}
\end{equation}

About this Ricker Model is known that in respect to the intrinsic growth rate $r$ the behavior changes 
from stable, over period doubling to chaos.  However we want to reduce this impact, therefore we use 
the Beverton-Holt map as a also well known model. This model is also a discrete-time equivalent to 
logistic growth, which shows stable behavior and approaches the carring capacity in every permitted parameter set. 

The Beverton-Holt Model ist given by the following equation
\begin{equation}
  \begin{split}
    g(N_t) = \frac{\lambda N_t}{1 + \frac{(\lambda -1)N_t}{K}}
  \end{split}
\end{equation}

with $\lambda > 0$ as the intrinsic growth rate and $K > 0$ as the capacity.

Assuming that the interaction between host and parasitoid is density-dependent with a Holling type
\Romannum{3} results:

\begin{equation}
  \begin{split}
    f(N_t, P_t) = 1 - e^{- \frac{N_{enc}}{N_t}} =1- e^{\left[- \frac{aTN_t P_t}{1+cN+aT_h N_t^2} \right]} 
  \end{split}
\end{equation}

with the total time per generation $T$, the handling time $T_h$ and the parasitization rate $a$.

Through dedimensionalization with
\begin{equation}
  n_t = \frac{N_t}{K} \hspace*{10mm} p_t=\frac{P_t}{bK} \hspace*{10mm} \alpha = abTK^2 \hspace*{10mm} \mu = aT_h K^2
\end{equation}
one receives:
\begin{equation}
  \begin{split}
    n_{t+1} &= \frac{\lambda n_t}{1+(\lambda -1) n_t} e^{\frac{-\alpha n_t p_t}{1-\mu n_t^2}}\\
    p_{t+1} &= n_t \left( 1-e^{\frac{-\alpha n_t p_t}{1+\mu n_t^2}}\right)
  \end{split}
\end{equation}
with the scaled growth rate $\lambda > 0$, the scaled parasitization rate $\alpha > 0$ and 
the scaled handling time $\mu > 0$



\subsection{The Algorithms}
The project was implemented in \href{https://docs.julialang.org/en/v1/}{Julia}\footnote[1]{https://docs.julialang.org/en/v1/} and uses the DrWatson package.
To reproduce the results look at README.md on \href{https://github.com/relias96/Studienprojekt}{github.com}\footnote[2]{https://github.com/relias96/Studienprojekt}

Because of the complex behavior the existance and stability of an Attractor is hardly 
analytical shown. Therefore we investigate this behavior by using a few numerical methods. 
We choose the Julia programming Language because many required functionalities are already implemented, 
with easy readable code and great performance.\par

\subsubsection{Orbit-diagramm}
An orbit-diagramm is often mistakenly called bifurcation-diagramm. One thing in common is that a parameter is varied over a range, but while an orbit-diagramm only 
shows attracting equilibria/cycles or possible locations of chaos, a bifurcation diagram in contrast plots all attracting and repelling periodic points \autocite{bifurcation}.

While the computation of the bifurcation diagram can be very complex and time-consuming, the computation of an orbit diagram is relatively simple. Since an orbit diagramm 
is sufficient, we will use it in the following. Because there are co-existing attractors, we implement a algorithm in the following manner:

\begin{algorithm}[H]
  \caption{Orbitdiagram}\label{orbit}
  \begin{algorithmic}[1]
  \Function{timeseries}{system, n, m}
    \State evolve \textit{system} \textit{m} times
    \State $\textit{ts} \gets \text{evolve system } \textit{n} \text{ times}$
    \State \Return \textit{ts}
  \EndFunction
  \Function{init}{system, i}
    \State $\textit{system.initialState} \gets \textit{i}$
  \EndFunction
  \Procedure{MyProcedure}{system, p\_values, period\_limit}
    \State $\textit{m} \gets \text{transient time}$
    \State $\textit{n} \gets \text{tracked steps}$
    \ForAll{$\textit{p} \in \text{p\_values}$}
      \Repeat
        \State \Call{init}{system, random}
        \State $attractor \gets Set(\Call{timeseries}{system, n, m})$
        \If{$length(attractor) < period\_limit$}
          \State $\text{\Call{Plot}{p,attractor} as dots}$
        \Else 
          \State $\text{\Call{Plot}{p,attractor} as blurred dots}$
        \EndIf
      \Until{$n > 10$}
    \EndFor
  \EndProcedure
  \end{algorithmic}
\end{algorithm}

\subsubsection{Basin of attraction}



\begin{figure}[H]
  \begin{centering}
    \includegraphics[width =\textwidth]{../plots/basinOfAttraction.png} 
    \caption{Basin of Attraction}\label{fig:basinAlgo}
  \end{centering}
\end{figure}

Figure \ref{fig:basinAlgo} shows the algorithm behind the function \texttt{basins_of_attraction} which relies on a method described in Chapter 7 of \autocite{nusse2012dynamics}.
To get the basin of attraction we use a finate state machine. The state space is devided into a discrete grid initialized with a value \texttt{unvisited}.
For each inital condition we track the trajectory of the System by using the mapper \texttt{AttractorsViaRecurrences} which numerically evolvs the given System using the \texttt{trajectory} function 
from the \texttt{DynamicalSystems} libary and tracks its path in the Statespace. Once an Attractor hits a cell the value 
changes to  \texttt{visited} (upper left of Fig:\ref*{fig:basinAlgo}). When the trajectory hits consecutive times previos \texttt{visited} cells an Attractor is detected (upper right of Fig:\ref*{fig:basinAlgo}).
In the following Step the new found Attraktor is tracked over \texttt{n} time itteration (lower left of Fig:\ref{fig:basinAlgo}). In the last Step the inital condition is mapped to the Attractor.
Following this algorithm each inital conditions can be matched with one of the Attractors to recieve the basin of attraction.

For a more detailed description see `Effortless estimation of basin of attraction' by Datseries \autocite{datseris}.

\section{Results}

By the fact, that we use De-dimensionalized Equations and that we do not have a real life scenario to fit this for, our goal is to gain a qualitative understanding of the Model.
Kaitala, Ylikarjula and Heino have shown that their Host-Parasitoid Model, under the assumption of Host groth following a Ricker map, behaves in some strage ways. Due to the behavior
of the Ricker map, which shows Choas at some parameter values, we choose the Beverton-Holt map with its stable Behavior over all parameters to eliminate this destabilizing impacts on
the overall behavior. Never the less our Model is highly nonlinear and has some interisting features.

\subsection{Overview}
The model shows the following complex behavior:
\begin{itemize}
  \item Supertransient behavior
  \item Intermittency
  \item Non-unique dynamics, which mean coexisting attractors
  \item Basin of attraction
\end{itemize}


\subsection{Transient behavior}
At first we discovered long transient phases before the Model reaches its final state. This means that it takes a unexpected long time of inconclusive behavior before the behavior settles
to a state. In Figure \ref{fig:transient} the Model ($\lambda=2.5$; $\alpha = 17.23 $; $ \mu = 0.1$) reaches it's Equilibrium after 8000 generations. 

\begin{figure}[H]
\begin{centering}
\includegraphics[width =\textwidth]{../plots/Project_Report/Supertransient.png}
\caption{Supertransient behavior}\label{fig:transient}
\end{centering}
\end{figure}

\subsection{Intermittency}

For some parameter combination intermittency occurs in our Model. 
This means that a equlibrium is no longer stable, but instead randomly switches between states. 
In figure \ref{fig:intermittency} the timeseries for the parameters $\lambda=6.5$; $\alpha = 21.411 $; $ \mu = 0.1$ is plotted.
It is rare to find in this model, but it is possible. 
\begin{figure}[H]
\begin{centering}
\includegraphics[width =\textwidth]{../plots/Project_Report/Intermittency.png}
\caption{intermittency}\label{fig:intermittency}
\end{centering}
\end{figure}


\subsection{Non-unique Dynamics}

To explore the dynamics we make use of the orbitdiagram to track all attractors given for a parameter combination and plots them in the given way (see Figure: \ref{fig:Orbitdiagram}).


\begin{figure}[H]
  \begin{centering}
    \includegraphics[width =\textwidth]{../plots/Project_Report/Orbit7.png} 
    \caption{Orbitdiagram}
    \label{fig:Orbitdiagram}
  \end{centering}
\end{figure}
%$\lambda=6.5$; $\alpha = 21.411 $; $ \mu = 0.1$



\begin{figure}[H]
  \begin{centering}
    \includegraphics[width =\textwidth]{../plots/Project_Report/Basin1.png}
    \caption{Basin of attraction: Coexisting attractors}
  \end{centering}
\end{figure}


\begin{figure}[H]
  \begin{centering}
    \includegraphics[width =\textwidth]{../plots/Project_Report/Basin_piece_Chaos.png}
    \caption{Basin of attraction: Piece Chaos}
  \end{centering}
\end{figure}

\section{Diskursion}

The given System is a fully deterministic Model. Therefore fractal properties do not directly influence 
the stability of an Attractor. But under respect of random fluctuations, mixed up basins with close proximaty 
of different Attractors may disturb the System. A metric on the basin based on the neighborhood could quantify this
insecurity. 

sad\autocite{paper}
asdaf\autocite{biodiversity}
asgs\autocite{hassell}

\newpage
\thispagestyle{justline}
\printbibliography[]


\end{document}